%%%%%%%%%%%%%%%%%%%%%%%%%%%%%%%%%%%%%%%%%
% Journal Article
% LaTeX Template
% Version 1.4 (15/5/16)
%
% This template has been downloaded from:
% http://www.LaTeXTemplates.com
%
% Original author:
% Frits Wenneker (http://www.howtotex.com) with extensive modifications by
% Vel (vel@LaTeXTemplates.com)
%
% License:
% CC BY-NC-SA 3.0 (http://creativecommons.org/licenses/by-nc-sa/3.0/)
%
%%%%%%%%%%%%%%%%%%%%%%%%%%%%%%%%%%%%%%%%%

%----------------------------------------------------------------------------------------
%	PACKAGES AND OTHER DOCUMENT CONFIGURATIONS
%----------------------------------------------------------------------------------------

\documentclass[twoside,twocolumn]{article}

\usepackage{blindtext} % Package to generate dummy text throughout this template 

\usepackage{hyperref}
\hypersetup{
    colorlinks=true,
    linkcolor=blue,
    filecolor=magenta,      
    urlcolor=cyan,
}

\usepackage[sc]{mathpazo} % Use the Palatino font
\usepackage[T1]{fontenc} % Use 8-bit encoding that has 256 glyphs
\linespread{1.05} % Line spacing - Palatino needs more space between lines
\usepackage{microtype} % Slightly tweak font spacing for aesthetics
\usepackage[utf8]{inputenc}

\usepackage[portuguese]{babel} % Language hyphenation and typographical rules

\usepackage[hmarginratio=1:1,top=32mm,columnsep=20pt]{geometry} % Document margins
\usepackage[hang, small,labelfont=bf,up,textfont=it,up]{caption} % Custom captions under/above floats in tables or figures
\usepackage{booktabs} % Horizontal rules in tables

\usepackage{lettrine} % The lettrine is the first enlarged letter at the beginning of the text
\usepackage{float}
\usepackage{enumitem} % Customized lists
\setlist[itemize]{noitemsep} % Make itemize lists more compact

\usepackage{abstract} % Allows abstract customization
\renewcommand{\abstractnamefont}{\normalfont\bfseries} % Set the "Abstract" text to bold
\renewcommand{\abstracttextfont}{\normalfont\small\itshape} % Set the abstract itself to small italic text
%\usepackage[shortlabels]{enumitem}
\usepackage{enumitem}
\usepackage{titlesec} % Allows customization of titles
\renewcommand\thesection{\Roman{section}} % Roman numerals for the sections
\renewcommand\thesubsection{\roman{subsection}} % roman numerals for subsections
\titleformat{\section}[block]{\large\scshape\centering}{\thesection.}{1em}{} % Change the look of the section titles
\titleformat{\subsection}[block]{\large}{\thesubsection.}{1em}{} % Change the look of the section titles

\usepackage{fancyhdr} % Headers and footers
\pagestyle{fancy} % All pages have headers and footers
\fancyhead{} % Blank out the default header
\fancyfoot{} % Blank out the default footer
\fancyhead[C]{MO443 $\bullet$ Abril 2019 $\bullet$ Relatório 01} % Custom header text
\fancyfoot[RO,LE]{\thepage} % Custom footer text

\usepackage{titling} % Customizing the title section

\usepackage{hyperref} % For hyperlinks in the PDF

\usepackage{graphicx}
\usepackage{subfigure}

\usepackage{mathtools}
\DeclarePairedDelimiter\floor{\lfloor}{\rfloor}
%----------------------------------------------------------------------------------------
%	TITLE SECTION
%----------------------------------------------------------------------------------------

\setlength{\droptitle}{-4\baselineskip} % Move the title up

\pretitle{\begin{center}\Huge\bfseries} % Article title formatting
\posttitle{\end{center}} % Article title closing formatting
\title{Relatório - Trabalho 01 \\ \Large MO443 - Introdução ao Processamento de Imagem Digital} %
%\subtitle{qsdqwdqwd} %Article title
\author{%
\textsc{Vinicius Teixeira de Melo - RA: 230223} \\[1ex] % Your name
\normalsize Universidade Estadual de Campinas \\ % Your institution
\normalsize \href{mailto:viniciusteixeira@liv.ic.unicamp.br}{viniciusteixeira@liv.ic.unicamp.br} % Your email address
%\and % Uncomment if 2 authors are required, duplicate these 4 lines if more
%\textsc{Jane Smith}\thanks{Corresponding author} \\[1ex] % Second author's name
%\normalsize University of Utah \\ % Second author's institution
%\normalsize \href{mailto:jane@smith.com}{jane@smith.com} % Second author's email address
}
\date{\today} % Leave empty to omit a date
\renewcommand{\maketitlehookd}{%
%\begin{abstract}
%\noindent \blindtext % Dummy abstract text - replace \blindtext with your abstract text
%\end{abstract}
}

%----------------------------------------------------------------------------------------

\begin{document}

% Print the title
\maketitle

%----------------------------------------------------------------------------------------
%	ARTICLE CONTENTS
%----------------------------------------------------------------------------------------

\section{Especificação do Problema}

O objetivo deste trabalho é implementar alguns filtros de imagens no domínio espacial e de frequências. A filtragem aplicada a uma imagem digital é uma operação local que altera os valores de intensidade dos pixels da imagem levando-se em conta tanto o valor do pixel em questão quanto valores de pixels vizinhos.

No processo de filtragem, utiliza-se uma operação de convolução de uma máscara pela imagem. Este processo equivale a percorrer toda a imagem alterando seus valores conforme os pesos da máscara e as intensidades da imagem.

O trabalho está dividido em duas seções principais:

\begin{itemize}
	\item Filtragem no Domínio Espacial
	\item Filtragem no Domínio de Frequências
\end{itemize}

Na seção de filtragem no domínio espacial, são dados 4 seguintes tipos de filtros:

\begin{enumerate}[label=(\alph*)]
\item $h_{1} = $ \begin{tabular}[H]{|c|c|c|c|c|}
\hline
0  & 0  & -1 & 0  & 0  \\ \hline
0  & -1 & -2 & -1 & 0  \\ \hline
-1 & -2 & 16 & -2 & -1 \\ \hline
0  & -1 & -2 & -1 & 0  \\ \hline
0  & 0  & -1 & 0  & 0  \\ \hline
\end{tabular}

\item $h_{2} = \frac{1}{256}$ \begin{tabular}{|c|c|c|c|c|}
\hline
1 & 4  & 6  & 4  & 1 \\ \hline
4 & 16 & 24 & 16 & 4 \\ \hline
6 & 24 & 36 & 24 & 6 \\ \hline
4 & 16 & 24 & 16 & 4 \\ \hline
1 & 4  & 6  & 4  & 1 \\ \hline
\end{tabular}

\item $h_{3} = $ \begin{tabular}{|c|c|c|}
\hline
-1 & 0 & 1 \\ \hline
-2 & 0 & 2 \\ \hline
-1 & 0 & 1 \\ \hline
\end{tabular}

\item $h_{4} = $ \begin{tabular}{|c|c|c|}
\hline
-1 & -2 & -1 \\ \hline
0  & 0  & 0  \\ \hline
1  & 2  & 1  \\ \hline
\end{tabular}
\end{enumerate}

As aplicações dos filtros devem ser feitas de forma individual, porém, os filtros $h_{3}$ e $h_{4}$ devem ser aplicados de forma combinada, somando-se as respostas de cada um dos filtros por meio da seguinte expressão: $\sqrt{(h_{3})^{2} + (h_{4}})^{2}$.

Na seção de filtragem no domínio de frequências, é necessário aplicar um filtro Gaussiano em uma imagem representada por seu espectro de Fourier, e a componente de frequência zero deve ser transladada para o centro do espectro. Nesse experimento, é necessário testar diferentes graus de suavização, de formaa borrar a imagem com mais ou menos intensidade.

%------------------------------------------------

\section{Entrada de Dados}

O código fonte criado para a execução de todas as tarefas está no notebook \textbf{Trabalho 01.ipynb}. O código foi criado para aceitar imagens em tons de cinza no formato RGB (\textit{Red, Green and Blue}) do tipo PNG (\textit{Portable Network Graphics}).

Para executar o notebook, basta iniciar o ambiente \textit{Jupyter Notebook}, abrir o notebook \textbf{Trabalho 01.ipynb} e executar as células em ordem. Todo o algoritmo foi implementado na linguagem Python na versão 3.6.

As imagens de entrada utilizadas nos testes do algoritmo foram retiradas da página do prof. Hélio Pedrini: \href{http://www.ic.unicamp.br/~helio/imagens_png/}{Imagens}. Na pasta \textbf{imgs/} estão as duas imagens monocromáticas utilizadas nos testes: \textbf{baboon.png} e \textbf{butterfly.png}. As dimensões das imagens de entrada utilizadas são 512x512.

\begin{figure}[tb]
\begin{center}
	\includegraphics[height=5cm]{figures/baboon.png}
\caption{baboon.png} \label{gdimotes}
\end{center}
\end{figure}

\begin{figure}[tb]
\begin{center}
	\includegraphics[height=5cm]{figures/butterfly.png}
\caption{butterfly.png} \label{gdimotes}
\end{center}
\end{figure}

%------------------------------------------------

\section{Código e Definições}

\subsection{Preenchimento de borda}

As imagens de entrada foram carregadas em memória utilizando uma função do \textbf{OpenCv} \cite{b1} que lê a imagem em escala de cinza e salva em uma estrutura $I_{m x n}$, na qual $m$ e $n$ representa a largura e a altura da imagem, respectivamente.

Para a aplicação da operação de convolução na imagem, foi necessário realizar um pré-processamento chamado de \textit{padding}. A operação de \textit{padding} é usada para gerar espaços em torno do conteúdo de um elemento, nesse caso, a imagem. Nesse trabalho, o \textit{padding} gerado nas imagens foi de tamanho $\floor*{\frac{x}{2}}$, onde $x$ é a dimensão do filtro que será convoluído com a imagem. Todo o \textit{padding} foi preenchido com o valor $0$, ou seja, com a cor preta.

\subsection{Convolução 2D}

Convolução 2D é uma operação que, a partir de imagem e um filtro, resulta em uma terceira imagem que mede a soma do produto da imagem e o filtro inicial ao longo da região subentendida pela superposição delas em função do deslocamento existente entre elas.

\begin{figure}[H]
\begin{center}
	\includegraphics[scale=.2]{figures/entrada.png}
	\includegraphics[scale=.2]{figures/filtro.png}
	\includegraphics[scale=.2]{figures/resultado.png}
\caption{Exemplo de imagem, filtro e resultado da convolução, respectivamente.} \label{gdimotes}
\end{center}
\end{figure}

A operação de convolução, primeiramente, rotaciona o filtro em $180^{\circ}$ e depois multiplica \textit{pixel} a \textit{pixel}, da seguinte forma:

\begin{figure}[H]
\begin{center}
	\includegraphics[scale=.2]{figures/mult1.png}
	\includegraphics[scale=.2]{figures/multi5.png}
	\includegraphics[scale=.2]{figures/mult9.png}
\caption{Exemplo de aplicação de convolução 2D.} \label{gdimotes}
\end{center}
\end{figure}

Nesse trabalho, foram utilizados 2 tipos de filtros:

\begin{enumerate}
	\item Filtro passa-alta: filtros utilizados para realçar certas características presentes nas imagens, tais como bordas, linhas ou regiões de interesse \cite{b2}.
	\item Filtro passa-baixa: filtros utilizados para suavizar as imagens, uma vez que as frequências altas correspondem às transições abruptas são atenuadas \cite{b2}.
\end{enumerate}

Os filtro $h_{1}$, $h_{3}$ e $h_{4}$ são exemplos de filtros passa-alta. Já o filtro $h_{2}$ é um exemplo de filtro passa-baixa. A seguir temos a exemplificação gráfica desses filtros.

\begin{figure}[H]
\begin{center}
	\includegraphics[scale=.3]{figures/h1.png}
	\includegraphics[scale=.3]{figures/h2.png}
\caption{Filtros $h_{1}$ e $h_{2}$.} \label{gdimotes}
\end{center}
\end{figure}

\begin{figure}[H]
\begin{center}
	\includegraphics[scale=.3]{figures/h3.png}
	\includegraphics[scale=.3]{figures/h4.png}
\caption{Filtros $h_{3}$ e $h_{4}$.} \label{gdimotes}
\end{center}
\end{figure}

\subsection{Redimensionamento}

A operação de redimensionamento foi utilizada para transformar as saídas da operação de convolução 2D, que não necessariamente estavam no intervalo de escalas de cinza de $\left[0,255\right]$, de volta ao intervalo de escalas de cinza representado por inteiros de 8 \textit{bits}.

\subsection{Transformada Discreta de Fourier}

A Transformada Discreta de Fourier (DFT) é uma transformação de coordenadas em componentes pertencentes ao conjunto dos números complexos, em que cada coeficiente é obtido pela combinação linear dos elementos da entrada com o núcleo da transformada \cite{b2}.

A Transformada de Fourier para duas variáveis pode ser escrita da seguinte forma:

\begin{equation*}
	F(u,v) = \int_{-\infty}^{\infty} \int_{-\infty}^{\infty} f(x,y)exp\left[-j2\pi(ux + vy\right] dx dy
\end{equation*}

e a partir de $F(u,v)$, pode-se obter $f(x,y)$ através da Transformada Inversa de Fourier:

\begin{equation*}
	f(x,y) =  \int_{-\infty}^{\infty} \int_{-\infty}^{\infty} F(u,v)exp\left[j2\pi(ux + vy)\right] du dv.
\end{equation*}

Na filtragem no domínio de frequência, o principal objetivo, é que as operações de convolução possuem o custo computacional menor do que no domínio espacial. Portanto, normalmente, o processamento de imagens possuem a seguinte sequência.

\begin{figure}[H]
\begin{center}
	\includegraphics[scale=.35]{figures/sequencia_fourier.png}
\caption{Sequência usualmente utilizada para as operações de processamento de imagens.} \label{gdimotes}
\end{center}
\end{figure}

\subsection{Filtro Gaussiano}

Os filtros Gaussianos são um tipo de filtro passa-baixa, os quais suavizam a imagem filtrada. Nos filtros Gaussianos, os coeficientes da máscara são derivados a partir de uma função Gaussiana bidimensional \cite{b2}. A função Gaussiana discreta com média zero e desvio padrão $\sigma$ é definida como

\begin{equation*}
	G(x,y) = \dfrac{1}{2\pi\sigma^{2}}exp\left(\dfrac{-(x^{2} + y^{2})}{2\sigma^{2}}\right).
\end{equation*}

As variações de sigma utilizados neste trabalho foram: $\sigma = \left[2,4,8,16,32,64\right]$.

\begin{figure}[H]
\begin{center}
	\includegraphics[scale=.2]{figures/filtro_gaussiano.png}
\caption{Exemplificação gráfica de um filtro Gaussiano.} \label{gdimotes}
\end{center}
\end{figure}

%------------------------------------------------

\section{Saída de Dados}

Os imagens resultantes da seção \textbf{1.1} e \textbf{1.2} foram salvas dentro da pasta \textbf{resultados/} utilizando uma função da biblioteca \textbf{matplotlib} chamada \textbf{matplotlib.pyplot.imsave()} \cite{b3}.

O formato dos nomes de saída estão da seguinte forma: para a seção \textbf{1.1} o padrão é o nome da imagem de entrada (\textit{baboon} ou \textit{butterfly}) concatenado com o nome do filtro aplicado sobre ela; na seção \textbf{1.2} 

%------------------------------------------------

\section{Resultados}

%----------------------------------------------------------------------------------------
%	REFERENCE LIST
%----------------------------------------------------------------------------------------

\begin{thebibliography}{99} % Bibliography - this is intentionally simple in this template

\bibitem{b1} Welcome to opencv documentation! \href{https://docs.opencv.org/2.4/index.html}{https://docs.opencv.org/2.4/index.html} Acesso em: 15/04/2019.

\bibitem{b2} Pedrini, Hélio, and William Robson Schwartz. Análise de imagens digitais: princípios, algoritmos e aplicações. Thomson Learning, 2008.

\bibitem{b3} Matplotlib Version 3.0.3 \href{https://matplotlib.org/contents.html}{https://matplotlib.org/contents.html} Acesso em: 15/04/2019.
 
\end{thebibliography}

%----------------------------------------------------------------------------------------

\end{document}
